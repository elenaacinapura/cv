%
% Simple asymmetric two-column CV 
% Author: Sofia JIJON
%

\documentclass[a4paper,10pt]{article}
\usepackage[vmargin=1.5cm, hmargin=1.0cm]{geometry}
\usepackage{setspace}
\input{MySetup}
%-------------------------------------------------------------------------------------------------------

\begin{document}
%-------------------------------------------------------------------------------------------------------
% Left column
%-------------------------------------------------------------------------------------------------------
\begin{adjustbox}{valign=t}
\begin{minipage}{0.3\textwidth} % Adapt width to your convenience
%----------------------------------------------------
% Please add a photo in 1x1 format
\begin{center}
\begin{tikzpicture}
	\clip (0,0) circle (2cm) node {\includegraphics[width=4cm]{propic}};
\end{tikzpicture}

\MySkip 	% See MySetup.tex file

%----------------------------------------------------
{\LARGE \bfseries Elena Acinapura}

\MySkip 	% See MySetup.tex file

Born in Bolzano, Italy, 1999\\
Living in Zürich, Switzerland\\

\MySkip 	% See MySetup.tex file

\textcolor{ColorTwo}{\faEnvelopeO} 
\myhref{mailto:elena.acinapura@gmail.com}{elena.acinapura@gmail.com} \\

\textcolor{ColorTwo}{\faChain} 
\myhref{https://elena.acinapura.it}{elena.acinapura.it}
\end{center}

\vfill

%----------------------------------------------------
\section*{Scientific interests}
\raggedright
\textcolor{ColorOne}{$\circ$} Quantum computing\\
\textcolor{ColorOne}{$\circ$} Experimental physics\\
\textcolor{ColorOne}{$\circ$} Computational physics\\

\vfill

%----------------------------------------------------
\section*{Education}
	\begin{description}
	\raggedright
	\item []\normalfont \textcolor{ColorTwo}{2021 - Now}\\
    \textbf{MSc. in Quantum Engineering}\\
	ETH Zürich\\
	Zürich -- Switzerland

	\item []\normalfont \textcolor{ColorTwo}{2018 - 2021}\\ \textbf{BSc. in
	Physics}\\
	University of Trento\\
	Trento -- Italy\\
    \small{Graduated with 110/110 cum laude}
    \normalsize

	\item []\normalfont \textcolor{ColorTwo}{2013 - 2018}\\
	\textbf{Secondary School}\\ 
	Liceo Classico \textit{"G. Carducci"} \\
	Bolzano -- Italy \\
    \small{Concluded with 100/100 cum laude} \normalsize
\end{description}

% \vfill
\end{minipage}
\end{adjustbox}
%
%
%-------------------------------------------------------------------------------------------------------
% Vertical rule
%-------------------------------------------------------------------------------------------------------
%
\hfill
\begin{adjustbox}{valign=t}
\begin{minipage}{0.05\textwidth} % Adapt width to your convenience
\MyVerticalRule  % See MySetup.tex file
\end{minipage}
\end{adjustbox}
% \hfill
%
%-------------------------------------------------------------------------------------------------------
% Right column
%-------------------------------------------------------------------------------------------------------
\begin{adjustbox}{valign=t}
\begin{minipage}{0.6\textwidth} % Adapt width to your convienience

%----------------------------------------------------
\section*{Working Experience}
\begin{description}
\setlength\itemsep{-1em}
\item[\normalfont \textcolor{ColorTwo}{Sep. 2022 -- Mar. 2023.}] 
	\textbf{Internship at Zurich Instruments}\\
	\emph{Zurich Instruments}
	\begin{spacing}{1.1}
		\small
	I am enabling automated system testing of the software interfacing with the Zürich Instruments hardware. I will also contribute to the testing capabilities by adding new tests covering complex experiments such as Repeat-Until-Success and Bell State Stabilization.
	\end{spacing}
\item[\normalfont \textcolor{ColorTwo}{Jul. 2022 -- Aug. 2022.}] 
	\textbf{Research assistant at the Photonic Lab}\\
	\emph{ETH Zurich, Supervisor: Prof.\ Dr.\ Lukas Novotny}
	\begin{spacing}{1.1}
		\small
	I continued the activities started in my semester project (see below), working on further improving the setup and solving noise-to-ratio related issues.
	\end{spacing}
\item[\normalfont \textcolor{ColorTwo}{Mar. 2022 -- Jun. 2022.}] 
	\textbf{Semester project at the Photonic Lab}\\
	\emph{ETH Zurich, Supervisor: Prof.\ Dr.\ Lukas Novotny}
	\begin{spacing}{1.2}
		\small
	I worked on the generation of a bistable trap with optical tweezers and studied the stochastic motion of a dielectric nanoparticle trapped in the bistable potential.
	\end{spacing}
\item[\normalfont \textcolor{ColorTwo}{Jul. 2021 -- Dec. 2021.}] 
	\textbf{Technical consultant}\\
	\emph{KERR s.r.l.}
	\begin{spacing}{1.1}
		\small
		I worked on the development of the front-end electronics for a single-photon detector.
	\end{spacing}
% \item[\normalfont \textcolor{ColorTwo}{Feb. 2021 -- Jun. 2021}] 
% 	\textbf{Physics Tutor}\\
% 	\emph{University of Trento}
% 	\begin{spacing}{1.1}
% 		\small 
% 		Physics I Teaching assistant at the department of Industrial Engineering.
% 	\end{spacing}
\end{description}
%----------------------------------------------------
\vspace{-1cm}
\section*{Knowledge and Technical Skills}
\begin{description}
\setlength\itemsep{-2em}

\item \textbf{Quantum Information Processing}
\begin{spacing}{1.1}
	\small Solid knowledge of the foundations of quantum optics, quantum algorithms, quantum error correction, superconducting circuits, ion traps, quantum dots and spin qubits.
\end{spacing}
\item \textbf{Computer Science}
\begin{spacing}{1.1}
	\small
	Experienced in algorithms, data structures and testing engineering.\\
	Main programming languages: 
	\begin{itemize}
		\item Python: very solid Knowledge; experience in data analysis for scientific experiments and test engineering with pytest.
		\item C/C++: very good knowledge; experience from the development of a library for computational physics as a personal project.
		% \item Git: used extensively both for personal and shared projects.
	\end{itemize}
\end{spacing}
\end{description}
%----------------------------------------------------
\vspace{-1cm}
\section*{Further Experiences}
\begin{description}

\raggedright
\item[\normalfont \textcolor{ColorTwo}{Spring 2022 - Now}] 
	\textbf{QEC Board member}\\
	\begin{spacing}{1.1}
		\small
	I am a board member of the Quantum Engineering Commission, the ETH students' association about quantum science and technologies.
	\end{spacing}

\raggedright
\item[\normalfont \textcolor{ColorTwo}{March 2021}] 
	\textbf{SWERC Competition}\\
	\begin{spacing}{1.1}
		\small
		I participated in SWERC, the international academic competition of algorithmic programming.
	\end{spacing}

\item[\normalfont \textcolor{ColorTwo}{February 2020 and 2021}] 
	\textbf{Google Hashcode}\\
	\begin{spacing}{1.1}
		\small
		I participated in the Google Hashcode competition.
	\end{spacing}
	
\end{description}

% \MySkip
% \LastUpdate
\end{minipage}
\end{adjustbox}
\end{document}